%!TEX root=../protocol.tex	% Optional

\section{Introduction}
Quantum computing is a technical field that has been advancing over the past decades. Beginning in 1980s, quantum computing evolved into a technology that big companies (Microsoft, IBM, Amazon) invest resources into, to further progress it. But what exactly is quantum computing and why does it gain more and more importance?
Quantum computers have some similarities and some differences to classical computers. The fundamental basis to this technology is the rules and principles of Quantum Theory, which explain the behavior of energy and matter on a quantum level. To understand how a quantum computer functions, a certain understanding of quantum physics has to be gained first. 
\subsection{Principles of the Quantum Theory}
To get a deep understanding of this technology, a basis of knowledge regarding quantum physics has to be established. The most important principles needed for quantum computing will be explained briefly.
\subsubsection{Superposition}
In 1927 Niels Bohr formulated the principles of Complementarity. It holds that there are pairs of complementary properties to objects that cannot be observed or measured simultaneously. Furthermore he states that properties of particles cannot be assumed to have a certain value, or even exist, until they are measured. This leads to a principle called 'Superposition'. Since a property cannot be assumed to have a specific value, we cannot know the actual state and it can be considered to hold all the values at once, as long as it is not measured. This is what is called  'being in Superposition'. An example for this would be the well known 'Schrödinger's Cat'. In this thought experiment, a cat is put into an enclosed box that contains a vial of deadly poison, some radioactive material and a geiger counter. The poison is released once the geiger counter detects radioactivity (i.e. when a single atom decays). It is impossible to know whether the cat has died or not, which can be seen as the cat being dead and alive simultaneously. The cat is in superposition until the box is opened and examined.
\subsubsection{Quantum entanglement}
Quantum entanglement describes a phenomenon in which quantum states of 2 or more particles are connected to each other. Measuring the state of one of the particles will provide information about the other and vice versa. This connection is not locally bound, it still remains even over great distances and happens instantaneously, which means that it is not limited by the speed of light. As an example, if two particles are entangled and one of them is measured to have an 'up' spin, the other one will have a 'down' spin. Information about a particle can be acquired without having to measure it, which can be used as an advantage in quantum computers.
\newpage
\subsection{General structure of quantum computers}
\subsubsection{Qubits}
In classical computing data is represented as so called bits (binary digits). Each bit can hold a value of either 1 (on) or 0 (off). When adding more and more bits, more data can be processed since more states can be created. Classical bits are usually represented as electric current controlled by a transistor. In quantum computing we have qubits (quantum binary digits) and they show both similar and unique characteristics. When measured, a qubit also provides a value of either 1 or 0. Though when not observed, qubits can be raised into superposition which assigns them a value of both 1 and 0. The probability of the measured outcome being an expected value can be manipulated. Furthermore qubits can be entangled with one another to influence each other in a logically useful way. These unique properties make it possible to create logical units that output the expected value.
In today's quantum computers, qubits are mostly made of electrons, photons or ions and the measured value mostly either equates to the spin or the polarization of the particle.
A number of qubits combined is called a quantum register. 
\subsubsection{Quantum gates}

Paper introducing Quantum Computation and fault tolerance \cite{2015arXiv150803695P}
\section{Current physical limitations}

\textit{For small scale quantum computers, it is sufficient to consider the five basic DiVincenzo criteria: ability to add qubits, highfidelity initialization and measurement, low decoherence, and a universal set of quantum gates. However, these criteria are insufficient for a large-scale quantum computer. DiVincenzo’s added two communications criteria — the ability to convert between stationary and mobile qubit representations, and to faithfully transport the mobile ones from one location to another and convert back to the stationary representation — are also critical, but so is gate speed ('clock rate'), the parallel execution of gates, the necessity for feasible large-scale classical control systems and feed-forward control, and the overriding issues of manufacturing, including the reproducibility of structures that affect key tuning parameters.} \cite{2009arXiv0906.2686V}

Papers focusing on quantum computer architecture  \cite{2012PhRvX...2c1007J}

Paper focusing on natural limitations of quantum computing
Deeper explanation of problems appearing in a physical context
\subsection{Decoherence}
Short overview on what quantum decoherence is, why it happens and why prevention is necessary
\subsection{Relaxation}
Short overview on quantum relaxation and how it affects quantum computing

\section{Techonological methods to prevent decoherence}

Information about fault tolerance \cite{2015arXiv150803695P}
\subsection{Quantum correcting code}

Information about quantum error correction \cite{2009arXiv0904.2557G}
\section{Approaches to solutions}
Proposing solutions to stated problems

Architectures minimizing errors \cite{2009arXiv0906.2686V} \cite{2012PhRvX...2c1007J}
\section{Future advancements}
How possible solutions might be implemented in the future and what might be possible/plausible.

\section{Conclusion}
Conclusion of paper and possibly answer to proposed question

Estimated hours of work: 12-14h

\nocite{*}